\documentclass[12pt]{article}
\usepackage{url, graphicx, epstopdf, amsmath, esint}
\usepackage{physics}

% page layout
\setlength{\topmargin}{-0.25in}
\setlength{\textheight}{9.5in}
\setlength{\headheight}{0in}
\setlength{\headsep}{0in}
\setlength{\parindent}{1.1\baselineskip}
\addtolength{\oddsidemargin}{-0.75in}
\setlength{\marginparwidth}{2in}

% problem formatting
\newcommand{\problemname}{Problem}
\newcounter{problem}
\newcommand{\startproblem}{\paragraph{Problem~\theproblem:}\refstepcounter{problem}}

% words
\newcommand{\foreign}[1]{\textsl{#1}}
\newcommand{\vs}{\foreign{vs}}

% math
\renewcommand{\vec}[1]{\boldsymbol{#1}}
% \newcommand{\dd}{\mathrm{d}} % PROVIDED IN physics PACKAGE
\newcommand{\e}{\mathrm{e}}
% \newcommand{\cross}{\times} % PROVIDED IN physics PACKAGE
% \newcommand{\curl}{\vec{\nabla}\times} % PROVIDED IN physics PACKAGE

% primary units
\newcommand{\rad}{\mathrm{rad}}
\newcommand{\kg}{\mathrm{kg}}
\newcommand{\m}{\mathrm{m}}
\newcommand{\s}{\mathrm{s}}
\newcommand{\A}{\mathrm{A}}

% secondary units
\renewcommand{\deg}{\mathrm{deg}}
\newcommand{\km}{\mathrm{km}}
\newcommand{\cm}{\mathrm{cm}}
\newcommand{\mm}{\mathrm{mm}}
\newcommand{\mum}{\mathrm{\mu m}}
\newcommand{\nm}{\mathrm{nm}}
\newcommand{\ft}{\mathrm{ft}}
\newcommand{\mi}{\mathrm{mi}}
\newcommand{\AU}{\mathrm{AU}}
\newcommand{\ns}{\mathrm{ns}}
\newcommand{\h}{\mathrm{h}}
\newcommand{\yr}{\mathrm{yr}}
\newcommand{\N}{\mathrm{N}}
\newcommand{\J}{\mathrm{J}}
\newcommand{\eV}{\mathrm{eV}}
\newcommand{\MeV}{\mathrm{MeV}}
\newcommand{\W}{\mathrm{W}}
\newcommand{\Pa}{\mathrm{Pa}}
\newcommand{\C}{\mathrm{C}}
\newcommand{\V}{\mathrm{V}}
\newcommand{\ohm}{\mathrm{\Omega}}
\newcommand{\muF}{\mathrm{\mu F}}
\newcommand{\Hz}{\mathrm{Hz}}
\newcommand{\GHz}{\mathrm{GHz}}

% derived units
\newcommand{\mps}{\m\,\s^{-1}}
\newcommand{\mph}{\mi\,\h^{-1}}
\newcommand{\mpss}{\m\,\s^{-2}}
\newcommand{\radps}{\rad\,\s^{-1}}

% random stuff
\sloppy\sloppypar\raggedbottom\frenchspacing\thispagestyle{empty}

\begin{document}

\section*{NYU Physics 2---Problem Set 12}

Due Thursday 2020 May 7 before lecture.

\paragraph{Problem~\theproblem:}\refstepcounter{problem}%
In class we considered a plane-parallel sinusoidal wave of the form
\begin{align}
  \vec{E} &= E_0\,\hat{x}\,\cos(k\,z - \omega t) \\
  \vec{B} &= B_0\,\hat{y}\,\cos(k\,z - \omega t) \quad ,
\end{align}
where $E_0$ and $B_0$ are amplitudes, $k$ is the magnitude of a
wave number, and $\omega$ is an angular frequency.
The $y$ component of the Maxwell's equation involving
$\curl{\vec{E}}$ in vacuum is
\begin{align}
  \frac{\dd}{\dd z}E_x - \frac{\dd}{\dd x}E_z = -\frac{\dd}{\dd t}B_y
  \quad ,
\end{align}
and the $x$ component of the equation involving $\curl{\vec{B}}$ is
\begin{align}
  \frac{\dd}{\dd y}B_z - \frac{\dd}{\dd z}B_y = \mu_0\,\epsilon_0\,\frac{\dd}{\dd t}E_x
  \quad .
\end{align}
Show---by taking the derivatives and plugging in---that the given forms
for the electric and magnetic fields can
satisfy these equations, provided that there is some particular relationship
between $E_0$ and $B_0$ and some particular relationship between $k$ and $\omega$.
What are those relationships?

You should be able to write an
expression for the ratio $E_0/B_0$ and an expression for the ratio $\omega/k$.
Both of these expressions should depend only on fundamental constants
$\mu_0$, $\epsilon_0$, and $c$.

\paragraph{Problem~\theproblem:}\refstepcounter{problem}%
In different wavelength regions, physicists tend to use different
units to describe photons. Make a table giving the wavelength
$\lambda$, wave number $k$, frequency $f$, angular frequency $\omega$,
and energy $E$ of each of the following photons:
\begin{itemize}
\item a radio wave at $f=70\,\GHz$ detected by ESA \textsl{Planck}
\item an infrared photon at $\lambda=10\,\mum$ emitted by Earth to space
\item a blue photon at $\lambda=410\,\nm$ absorbed by a leaf in photosynthesis
\item an x-ray at $E=1.02\,\MeV$ emitted by electron--positron annihilation
\end{itemize}
For the energies $E$ you will have to use that a photon has energy $E=h\,f$
where $h$ is the Planck constant. Give your answers in SI units except for the
energies, which you should give in $\eV$.

\paragraph{Problem~\theproblem:}\refstepcounter{problem}%
$\vec{E}$ and $\vec{B}$ fields in a charging and discharging capacitor.
Pointing flux?

\paragraph{Problem~\theproblem:}\refstepcounter{problem}%
Circular polarization problem? Standing-wave problem?

\end{document}
