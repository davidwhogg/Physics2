\documentclass[12pt]{article}
\usepackage{url, graphicx, epstopdf, amsmath, esint}
\usepackage{physics}

% page layout
\setlength{\topmargin}{-0.25in}
\setlength{\textheight}{9.5in}
\setlength{\headheight}{0in}
\setlength{\headsep}{0in}
\setlength{\parindent}{1.1\baselineskip}
\addtolength{\oddsidemargin}{-0.75in}
\setlength{\marginparwidth}{2in}

% problem formatting
\newcommand{\problemname}{Problem}
\newcounter{problem}
\newcommand{\startproblem}{\paragraph{Problem~\theproblem:}\refstepcounter{problem}}

% words
\newcommand{\foreign}[1]{\textsl{#1}}
\newcommand{\vs}{\foreign{vs}}

% math
\renewcommand{\vec}[1]{\boldsymbol{#1}}
% \newcommand{\dd}{\mathrm{d}} % PROVIDED IN physics PACKAGE
\newcommand{\e}{\mathrm{e}}
% \newcommand{\cross}{\times} % PROVIDED IN physics PACKAGE
% \newcommand{\curl}{\vec{\nabla}\times} % PROVIDED IN physics PACKAGE

% primary units
\newcommand{\rad}{\mathrm{rad}}
\newcommand{\kg}{\mathrm{kg}}
\newcommand{\m}{\mathrm{m}}
\newcommand{\s}{\mathrm{s}}
\newcommand{\A}{\mathrm{A}}

% secondary units
\renewcommand{\deg}{\mathrm{deg}}
\newcommand{\km}{\mathrm{km}}
\newcommand{\cm}{\mathrm{cm}}
\newcommand{\mm}{\mathrm{mm}}
\newcommand{\mum}{\mathrm{\mu m}}
\newcommand{\nm}{\mathrm{nm}}
\newcommand{\ft}{\mathrm{ft}}
\newcommand{\mi}{\mathrm{mi}}
\newcommand{\AU}{\mathrm{AU}}
\newcommand{\ns}{\mathrm{ns}}
\newcommand{\h}{\mathrm{h}}
\newcommand{\yr}{\mathrm{yr}}
\newcommand{\N}{\mathrm{N}}
\newcommand{\J}{\mathrm{J}}
\newcommand{\eV}{\mathrm{eV}}
\newcommand{\MeV}{\mathrm{MeV}}
\newcommand{\W}{\mathrm{W}}
\newcommand{\Pa}{\mathrm{Pa}}
\newcommand{\C}{\mathrm{C}}
\newcommand{\V}{\mathrm{V}}
\newcommand{\ohm}{\mathrm{\Omega}}
\newcommand{\muF}{\mathrm{\mu F}}
\newcommand{\Hz}{\mathrm{Hz}}
\newcommand{\GHz}{\mathrm{GHz}}

% derived units
\newcommand{\mps}{\m\,\s^{-1}}
\newcommand{\mph}{\mi\,\h^{-1}}
\newcommand{\mpss}{\m\,\s^{-2}}
\newcommand{\radps}{\rad\,\s^{-1}}

% random stuff
\sloppy\sloppypar\raggedbottom\frenchspacing\thispagestyle{empty}

\begin{document}

\section*{NYU Physics 2---Problem Set 3}

Due Thursday 2022 February 17 on Brightspace before 12:30\,pm.

\startproblem%
Take a look at a standard 9-volt battery.
Roughly what is the magnitude of the electric field in the space
between the terminals of this battery? Give a numerical answer in
SI units.
Be careful about the spacing or separation of the terminals.
What separation matters?

\startproblem%
A capacitor is made of two thin plates of very large area $A$
separated by a small distance $h$. The distance $h$ is small in the sense
that $h\ll \sqrt{A}$. If you put charges $+Q$ and $-Q$ on the
two plates, what is the potential difference $V$ between the plates?
Use the infinite-sheet approximation to compute the electric field.
Give your answer in terms of $A, h, Q, \epsilon_0$.

\startproblem%
Repeat the previous problem, but for a capacitor that is a very long
cylinder of length $L$, with the one ``plate'' being an inner thin
cylindrical shell of radius $a$, and the other plate being an outer
thin cylindrical shell of radius $b$. When I say ``long'' I mean both that
$L\gg b$ and also that you
can compute the field inside the capacitor assuming perfect
cylindrical symmetry everywhere. Show that when $a$ is very close to $b$ (that
is, when $a \equiv b - h$ and $h \ll b$), your answer becomes the same
as the answer to the previous problem.

\startproblem%
How much work does it take to charge up a thin spherical shell?
Consider a thin spherical shell of radius $a$, charged to a charge
$q$ (charge equally distributed over its surface). How much work
$\dd W$ does it take to add a tiny bit of charge $\dd q$, brought
from infinity? Now integrate $q$ from $0$ to $Q$ to get the total
work done to charge the shell slowly to full charge $Q$. At the
end of that charging, what is the potential difference $V$ between
the shell surface and infinity?

\end{document}
