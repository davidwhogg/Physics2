\documentclass{article}
\begin{document}
\thispagestyle{empty}
\section*{NYU Physics 2 --- Problem Set 4}

\emph{Due by Friday 2002 February 22 at 1pm at Irene Port's office in
Meyer 424.}

\subsection*{Problem 1}

Estimate the capacitance, in $\mathrm{\mu F}$, of a storm cloud over
Lake Ontario.  Treat the system as a parallel-plate capacitor with the
area of the lake and a separation of $1000~\mathrm{ft}$.  How much
energy is stored in the capacitor when lightning strikes?  To compute
this, you will have to use the potential difference associated with
the breakdown of air over a distance of $1000~\mathrm{ft}$.  Give your
answer in $\mathrm{J}$ and in kilotons of TNT equivalent, where 1
kiloton is about $4\times 10^{12}~\mathrm{J}$.

\subsection*{Problem 2}

Compute the capacitance of a cylindrical capacitor in which the inner
plate is a cylinder of radius $a$ and the outer plate is a cylinder of
radius $b>a$, both of length $\ell$.  Compute the capacitance by
(a)~explicitly computing the electric field magnitude $E(r)$ as a
function of cylindrical radius $r$ for a charges of $+Q$ and $-Q$ on
the plates, and (b)~explicitly integrating the field to get the
potential difference $V$.  Take the limit in which $(b-a)\ll a$, and
show that the capacitance you get is the same as a ``rolled-up''
parallel-plate capacitor.

\subsection*{Problem 3}

Integrate the energy density $\epsilon_0\,|\vec{E}|^2/2$ over the
volume in the cylindrical capacitor of Problem~2, and show that the
total energy stored in the field is equal to $Q^2/(2\,C)$.  Do
\emph{not} assume that $(b-a)\ll a$.  This problem involves setting up
a non-trivial integral; show all your steps (because setting up the
integral is much more important than integrating it).  What is the
energy, in $\mathrm{J}$, in a piece of coaxial cable (like the one connected to
your television) of length $1~\mathrm{m}$, inner radius
$1~\mathrm{mm}$, and outer radius $2~\mathrm{mm}$, charged up to
$120~\mathrm{V}$?

\end{document}
