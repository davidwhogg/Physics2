\documentclass{article}
\begin{document}
\thispagestyle{empty}
\section*{NYU Physics 2 --- Problem Set 9}

\emph{Due by Friday 2002 April 5 at 1pm at Irene Port's office in
Meyer 424.}

\subsection*{Problem 1}

Imagine that 120~V AC electricity is generated with a single loop of
area $1~\mathrm{m^2}$ rotating at 60~rps in a magnetic field.  What is
the required strength of the magnetic field?  State your assumptions
and show your work.  How does your answer change if the loop has 1000
windings?

\subsection*{Problem 2}

A toroidal solenoid of inner radius $a$, outer radius $b$, and height
$h$ has $N$ equally spaced turns of wire around it.  If this solenoid
carries current $I$, what is the magnetic field everywhere inside the
solenoid?  Use Ampere's law and assume cylindrical symmetry (not a
terrible assumption, it turns out).  Now imagine that this solenoid is
pierced though its center, along the axis of symmetry, by a very long,
straight wire.  What is the mutual inductance between the wire and the
solenoid?  Draw diagrams to explain your reasoning and your answer.
\emph{Pay attention in lecture for some hints on this problem.}

\subsection*{Problem 3}

A square loop of side length $a$, mass $m$, and total resistance $R$,
aligned with the $x$--$z$ plane, falls vertically in the negative $z$
direction under the influence of gravity $\vec{g}=-g\,\hat{k}$.  Space
is filled with a $y$-direction magnetic field $\vec{B}=B_y\,\hat{j}$
the magnitude of which varies with height as
\begin{equation}
B_y= B_0+\alpha\,z \;\;\; .
\end{equation}
What is the terminal velocity $v_z$ of the loop?

\end{document}
