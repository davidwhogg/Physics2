\documentclass[12pt]{article}
\usepackage{url, graphicx, epstopdf, amsmath, esint}
\usepackage{physics}

% page layout
\setlength{\topmargin}{-0.25in}
\setlength{\textheight}{9.5in}
\setlength{\headheight}{0in}
\setlength{\headsep}{0in}
\setlength{\parindent}{1.1\baselineskip}
\addtolength{\oddsidemargin}{-0.75in}
\setlength{\marginparwidth}{2in}

% problem formatting
\newcommand{\problemname}{Problem}
\newcounter{problem}
\newcommand{\startproblem}{\paragraph{Problem~\theproblem:}\refstepcounter{problem}}

% words
\newcommand{\foreign}[1]{\textsl{#1}}
\newcommand{\vs}{\foreign{vs}}

% math
\renewcommand{\vec}[1]{\boldsymbol{#1}}
% \newcommand{\dd}{\mathrm{d}} % PROVIDED IN physics PACKAGE
\newcommand{\e}{\mathrm{e}}
% \newcommand{\cross}{\times} % PROVIDED IN physics PACKAGE
% \newcommand{\curl}{\vec{\nabla}\times} % PROVIDED IN physics PACKAGE

% primary units
\newcommand{\rad}{\mathrm{rad}}
\newcommand{\kg}{\mathrm{kg}}
\newcommand{\m}{\mathrm{m}}
\newcommand{\s}{\mathrm{s}}
\newcommand{\A}{\mathrm{A}}

% secondary units
\renewcommand{\deg}{\mathrm{deg}}
\newcommand{\km}{\mathrm{km}}
\newcommand{\cm}{\mathrm{cm}}
\newcommand{\mm}{\mathrm{mm}}
\newcommand{\mum}{\mathrm{\mu m}}
\newcommand{\nm}{\mathrm{nm}}
\newcommand{\ft}{\mathrm{ft}}
\newcommand{\mi}{\mathrm{mi}}
\newcommand{\AU}{\mathrm{AU}}
\newcommand{\ns}{\mathrm{ns}}
\newcommand{\h}{\mathrm{h}}
\newcommand{\yr}{\mathrm{yr}}
\newcommand{\N}{\mathrm{N}}
\newcommand{\J}{\mathrm{J}}
\newcommand{\eV}{\mathrm{eV}}
\newcommand{\MeV}{\mathrm{MeV}}
\newcommand{\W}{\mathrm{W}}
\newcommand{\Pa}{\mathrm{Pa}}
\newcommand{\C}{\mathrm{C}}
\newcommand{\V}{\mathrm{V}}
\newcommand{\ohm}{\mathrm{\Omega}}
\newcommand{\muF}{\mathrm{\mu F}}
\newcommand{\Hz}{\mathrm{Hz}}
\newcommand{\GHz}{\mathrm{GHz}}

% derived units
\newcommand{\mps}{\m\,\s^{-1}}
\newcommand{\mph}{\mi\,\h^{-1}}
\newcommand{\mpss}{\m\,\s^{-2}}
\newcommand{\radps}{\rad\,\s^{-1}}

% random stuff
\sloppy\sloppypar\raggedbottom\frenchspacing\thispagestyle{empty}

\begin{document}

\section*{NYU Physics 2---Problem Set 10}

Due Thursday 2020 April 23 before lecture.

\paragraph{Problem~\theproblem:}\refstepcounter{problem}%
Look at the ``hello world'' AC circuits shown in Lectures on
2020-04-14 and 2020-04-16. For each of them, draw the circuit, showing
the sign convention you are using for current and voltage. And then
make time-aligned plots of the voltage \foreign{vs} time, current
\foreign{vs} time for two periods of oscillation, and power dissipated
$I\,V$ (the product). Make these plots for at least two periods of
oscillation. Clearly mark each plot with the maximum and minimum
values of the curves. These values will depend on variables like $V,
R, C, L, \omega$.

\paragraph{Problem~\theproblem:}\refstepcounter{problem}%
If you have an inductor $L$ and a capacitor $C$ in series, what does
the total complex impedance $Z$ approach as $\omega\rightarrow
(L\,C)^{-1/2}$? If you have the same two elements in parallel, what
does the total complex impedance approach in this same limit?

Now imagine that associated with the inductor $L$ there is a
resistance $R$. That is, there is always a small resistance $R$ in
parallel with the inductor, in both cases. Now what do your two limits
become?

\paragraph{Problem~\theproblem:}\refstepcounter{problem}%
In Lecture on 2020-04-16 we found, in the complex formalism, that for
an LRC circuit driven by an AC voltage source producing amplitude $V$ and frequency $\omega$, the current flowing in the
circuit is
\begin{equation}\label{eq:res}
  I = \frac{V}{R + i\,\left(\omega\,L - \frac{1}{\omega\,C}\right)}\,\exp(i\,\omega\,t)
  \quad .
\end{equation}
Plot the amplitude (the magnitude of the complex prefactor in front of
the exponential) of this current as a function of frequency
$\omega$. Be sure to label the plot with the maximum value, the limit
at $\omega\rightarrow 0$, and the asymptotic limit at
$\omega\rightarrow\infty$.

\paragraph{Problem~\theproblem:}\refstepcounter{problem}%
Here's a dumb model of an FM radio circuit: It is a circuit that has a
resonance (big current $I$ given driving voltage $V$) at the radio-station
carrier frequency $f$, but very little response (small current given
driving voltage) at the carrier frequency of other nearby radio stations.
Assume that the response is given by something like
equation~(\ref{eq:res}). What do you need to be true about $R$, $L$,
and $C$ for a radio to be tuned to the frequency of The Beat of New
York KTU at 103.5 (what units are those?),
but have less than 10-percent as much
response to a pirate radio station operating at 103.7? Assume, for
definiteness, that these two stations are impinging on the radio with
the same voltage amplitude $V$.

You are going to get one relationship between $L$ and
$C$ to ensure that the radio is tuned to 103.5, and then there is a
relationship also involving $R$ to ensure that the radio is
insensitive to 103.7.

\end{document}
