\documentclass[12pt]{article}
\usepackage{url, graphicx, epstopdf, amsmath, esint}
\usepackage{physics}

% page layout
\setlength{\topmargin}{-0.25in}
\setlength{\textheight}{9.5in}
\setlength{\headheight}{0in}
\setlength{\headsep}{0in}
\setlength{\parindent}{1.1\baselineskip}
\addtolength{\oddsidemargin}{-0.75in}
\setlength{\marginparwidth}{2in}

% problem formatting
\newcommand{\problemname}{Problem}
\newcounter{problem}
\newcommand{\startproblem}{\paragraph{Problem~\theproblem:}\refstepcounter{problem}}

% words
\newcommand{\foreign}[1]{\textsl{#1}}
\newcommand{\vs}{\foreign{vs}}

% math
\renewcommand{\vec}[1]{\boldsymbol{#1}}
% \newcommand{\dd}{\mathrm{d}} % PROVIDED IN physics PACKAGE
\newcommand{\e}{\mathrm{e}}
% \newcommand{\cross}{\times} % PROVIDED IN physics PACKAGE
% \newcommand{\curl}{\vec{\nabla}\times} % PROVIDED IN physics PACKAGE

% primary units
\newcommand{\rad}{\mathrm{rad}}
\newcommand{\kg}{\mathrm{kg}}
\newcommand{\m}{\mathrm{m}}
\newcommand{\s}{\mathrm{s}}
\newcommand{\A}{\mathrm{A}}

% secondary units
\renewcommand{\deg}{\mathrm{deg}}
\newcommand{\km}{\mathrm{km}}
\newcommand{\cm}{\mathrm{cm}}
\newcommand{\mm}{\mathrm{mm}}
\newcommand{\mum}{\mathrm{\mu m}}
\newcommand{\nm}{\mathrm{nm}}
\newcommand{\ft}{\mathrm{ft}}
\newcommand{\mi}{\mathrm{mi}}
\newcommand{\AU}{\mathrm{AU}}
\newcommand{\ns}{\mathrm{ns}}
\newcommand{\h}{\mathrm{h}}
\newcommand{\yr}{\mathrm{yr}}
\newcommand{\N}{\mathrm{N}}
\newcommand{\J}{\mathrm{J}}
\newcommand{\eV}{\mathrm{eV}}
\newcommand{\MeV}{\mathrm{MeV}}
\newcommand{\W}{\mathrm{W}}
\newcommand{\Pa}{\mathrm{Pa}}
\newcommand{\C}{\mathrm{C}}
\newcommand{\V}{\mathrm{V}}
\newcommand{\ohm}{\mathrm{\Omega}}
\newcommand{\muF}{\mathrm{\mu F}}
\newcommand{\Hz}{\mathrm{Hz}}
\newcommand{\GHz}{\mathrm{GHz}}

% derived units
\newcommand{\mps}{\m\,\s^{-1}}
\newcommand{\mph}{\mi\,\h^{-1}}
\newcommand{\mpss}{\m\,\s^{-2}}
\newcommand{\radps}{\rad\,\s^{-1}}

% random stuff
\sloppy\sloppypar\raggedbottom\frenchspacing\thispagestyle{empty}

\begin{document}

\section*{NYU Physics 2---Problem Set 10}

Due Thursday 2023 April 20 by 12:30\,pm on Brightspace.

\startproblem%
Look at the AC circuit we discussed in Lecture that had an AC generator $V,\omega$, a resistor $R$,
a capacitor $C$, and an inductor $L$, all in parallel.
The AC generator produces a time-dependent voltage
\begin{equation}
  V(t) = V_0\,\cos(\omega\,t) = \Re\left\{V_0\,\exp(i\,\omega\,t)\right\} ~,
\end{equation}
and it led to sinusoidally varying currents $I_R(t), I_C(t), I_L(t)$ in each of the three
parallel legs, some of which had $\cos(\omega\,t)$ and some of which had
$\sin(\omega\,t)$ dependence.
State a clear argument that each of these currents can be expressed in the
form
\begin{equation}
  I(t) = \Re\left\{\frac{V_0}{Z}\,\exp(i\,\omega\,t)\right\} ~,
\end{equation}
if we we associate an impedance to each of the circuit elements as follows:
\begin{align}
  Z_R &= R \\
  Z_C &= \frac{1}{i\,\omega\,C} \\
  Z_L &= i\,\omega\,L ~.
\end{align}
At some point you might need to use the facts $i^2 = -1$ and $1/i = -i$.

\startproblem%
If you have an inductor $L$ and a capacitor $C$ in series, what does
the total complex impedance $Z$ approach as $\omega\rightarrow
(L\,C)^{-1/2}$? If you have the same two elements in parallel, what
does the total complex impedance approach in this same limit?

Now imagine that, associated with the inductor $L$, there is a small
resistance $R$. That is, there is always a small resistance $R$ in
series with the inductor, in both cases. Now what do your two limits
become?

\textsl{Note:} To solve this problem you will have to use the fact that
complex impedances add in series and in parallel the way resistances do, but
now using complex math.

\startproblem%
In Lecture we will find (and in your textbook you will find), in the complex formalism, that for
an LRC circuit driven by an AC voltage source producing amplitude $V$ and frequency $\omega$, the current flowing in the
circuit is
\begin{equation}\label{eq:res}
  I = \frac{V}{R + i\,\left(\omega\,L - \frac{1}{\omega\,C}\right)}\,\exp(i\,\omega\,t)
  \quad .
\end{equation}
(Or, really, $I(t)$ is the real part of this expression!)
Plot the amplitude (the magnitude of the complex prefactor in front of
the exponential) of this current as a function of frequency
$\omega$. Be sure to label the plot with the maximum value, the behavior as
$\omega\rightarrow 0$, and the behavior as
$\omega\rightarrow\infty$.

\startproblem%
Here's a dumb model of an FM radio receiver circuit: It is a circuit that has a
resonance (big current $I$ given driving voltage $V$) at the radio-station
carrier frequency $f$, but very little response (small current given
driving voltage) at the carrier frequency of other nearby radio stations.
Assume that the response is given by something like what's given in (\ref{eq:res}).
What do you need to be true about $R$, $L$,
and $C$ for a radio to be tuned to the frequency of The Beat of New
York KTU at 103.5 (what units are those?),
but have less than 10-percent as much
response to a pirate radio station operating at 103.7? Assume, for
definiteness, that these two stations are impinging on the radio with
the same voltage amplitude $V$.

You are going to get one relationship between $L$ and
$C$ to ensure that the radio is tuned to 103.5, and then there is a
relationship also involving $R$ to ensure that the radio is
insensitive to 103.7.

\textsl{Notes:} Be careful with your units! Also, frequencies are in Hz
(cycles per second) but the equation above uses $\omega$ (radians per second).

\end{document}
