\documentclass{article}
\begin{document}
\thispagestyle{empty}
\section*{NYU Physics 2 --- Problem Set 13}

\emph{Due by Friday 2002 May 3 at 1pm at Irene Port's office in
Meyer 424.}

\subsection*{Problem 1}

On a sunny day, the typical insolation (power per unit area from the
Sun) is roughly $1000~\mathrm{W\,m^{-2}}$.  What is the rms average
electric field and what is the rms average magnetic field in this
Solar radiation?

\subsection*{Problem 2}

A current $I$ flows into a parallel-plate capacitor with circular
plates of radius $a$ (area $\pi\,a^2$) and separation $h\ll a$, so
that the electric field magnitude is increasing inside the capacitor
as a function of time.  Use Maxwell's equations and an assumption of
perfect cylindrical symmetry to determine the magnetic field $\vec{B}$
(magnitude and direction) everywhere.  Be sure to compute $\vec{B}$
for radii larger than $a$ as well as smaller.  Draw diagrams to show
the field configuration, and plot the magnitude of the field $\vec{B}$
as a function of radius.  Now do the same, but for a straight,
cylindrical wire of radius $a$, carrying current $I$ uniformly
distributed across the cross-sectional area of the wire.  Again, be
sure to compute the magnetic field for regions both inside and outside
the wire.

\subsection*{Problem 3}

Consider a plane electromagnetic wave traveling in the $+x$ direction
with electric field amplitude $E_0\,\hat{y}$, and another traveling in
the opposite direction with the same electric field amplitude.  Write
an expression for the vector electric field and vector magnetic field
of the superimposed sum of these waves, as a function of position and
time.  Use cosine and sine addition formulae to simplify your result.
What is the total energy density from the electric and magnetic fields
as a function of position and time?

\end{document}
