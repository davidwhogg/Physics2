\documentclass{article}
\begin{document}
\thispagestyle{empty}
\section*{NYU Physics 2 --- Problem Set 12}

\emph{Due by Friday 2002 April 26 at 1pm at Irene Port's office in
Meyer 424.}

\subsection*{Problem 1}

A circuit consists of an AC voltage source with amplitude ${\cal E}_0$
and frequency $\omega$, a resistor $R$, an inductor $L$, and a
capacitor $C$, all connected in series.  What is the time-averaged
power $\left<P_R\right>$ dissipated in the resistor as a function of
frequency?  Plot this function against frequency.  What are the
driving frequencies $\omega_1$, $\omega_2$ at which the dissipated
power is half the power dissipation at the maximum?

\subsection*{Problem 2}

The peak in the response of an $LRC$ circuit is at the natural
frequency $\omega_0=(LC)^{1/2}$.  The ``bandwidth'' of that response,
$\Delta\omega$, is roughly the inverse of the damping time $L/R$, as
you (ought to have) found in the previous problem.  If your car radio
has a capacitance of $1~\mathrm{nF}$, and you are tuned to WNYC at
AM~820, what is the inductance of the receiver, and what resistance do
you need to keep from hearing a ``superposition'' of WNYC and the next
station, at AM~830?  You will have to find out the units of AM
frequencies, and don't forget your factors of $2\pi$!  Is your limit
on the resistance an upper limit or a lower limit?

\subsection*{Problem 3}

A circuit consists of an AC voltage source with amplitude ${\cal E}_0$
and frequency $\omega$, a resistor $R$, and an inductor $L$, all
connected in series.  What is the phase lag $\phi$ of the current
relative to the driving voltage as a function of frequency?  Do your
answers make sense in the limit of $\omega\rightarrow 0$ and
$\omega\rightarrow\infty$?  Explain.

\end{document}
