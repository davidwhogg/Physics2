\documentclass[12pt]{article}
\usepackage{graphics}
\begin{document}

\section*{NYU Physics 2---In-class Exam 2}

\vfill

\paragraph{Name:} ~
\paragraph{email:} ~

\vfill
\noindent
\resizebox{\textwidth}{!}{\newcommand{\vspacer}{\rule[-2.75ex]{0pt}{7ex}}%
\newcommand{\iint}{\makebox[0.5ex][l]{$\displaystyle\int$}\int}%
\newcommand{\oiint}{\makebox[-0.20ex][l]{$\bigcirc$}\iint}%
\begin{tabular}{|c|c|c|}%
\hline\vspacer
$\displaystyle \vec{F}= q\,(\vec{E}+\vec{v}\times\vec{B})$
&
$\displaystyle
  \oiint \vec{E}\cdot\hat{n}\,\mathrm{d}A= \frac{Q_\mathrm{encl}}{\epsilon_0}$
&
$\displaystyle
  \vec{F}_{12}= \frac{k\,q_1\,q_2}{|\vec{r}_1-\vec{r}_2|^3}
    \,(\vec{r}_1-\vec{r}_2)$
\\\hline\vspacer
$\displaystyle \frac{1}{4\pi\,\epsilon_0}= k=
   9\times 10^9~\mathrm{\frac{N\,m^2}{C^2}}$
&
$\displaystyle V_{ab}= -\int_a^b \vec{E}\cdot\mathrm{d}\vec{r}= \phi_b-\phi_a$
&
$\displaystyle W= q\,V$
\\\hline\vspacer
$\displaystyle \phi(\vec{r})= \frac{k\,q}{|\vec{r}|}$
&
$\displaystyle C\equiv \frac{Q}{V}$
&
$\displaystyle U= \frac{1}{2}\,C\,V^2$
\\\hline\vspacer
$\displaystyle
   u= \frac{\epsilon_0}{2}\,|\vec{E}|^2+\frac{1}{2\,\mu_0}\,|\vec{B}|^2$
&
$\displaystyle \vec{E}_\mathrm{inside}\approx\frac{1}{\kappa}\,\vec{E}_\mathrm{vacuum}$
&
$\displaystyle R= \frac{\rho\,\ell}{A}$
\\\hline\vspacer
$\displaystyle V= I\,R$
&
$\displaystyle P= I\,V$
&
$\displaystyle \mathrm{d}\vec{F}= I\,\mathrm{d}\vec{\ell}\times\vec{B}$
\\\hline\vspacer
$\displaystyle \mathrm{d}\vec{B}= \frac{\mu_0}{4\pi}\,\frac{I\,\mathrm{d}\vec{\ell}\times\hat{r}}{r^2}$
&
$\displaystyle \oint \vec{B}\cdot\mathrm{d}\vec{r}= \mu_0\,I
   +\mu_0\,\epsilon_0\,\frac{\mathrm{d}}{\mathrm{d}t}
   \iint\vec{E}\cdot\hat{n}\,\mathrm{d}A$
&
$\displaystyle \mu_0= 4\pi\times 10^{-7}~\mathrm{T\,m\,A^{-1}}$
\\\hline\vspacer
$\displaystyle
  \oiint\vec{B}\cdot\hat{n}\,\mathrm{d}A= 0$
&
$\displaystyle \oint \vec{E}\cdot\mathrm{d}\vec{r}=
   -\frac{\mathrm{d}}{\mathrm{d}t}
   \iint\vec{B}\cdot\hat{n}\,\mathrm{d}A$
&
$\displaystyle V= -L\,\frac{\mathrm{d}I}{\mathrm{d}t}$
\\\hline\vspacer
$\displaystyle U= \frac{1}{2}\,L\,I^2$
&
$\displaystyle |\tilde{Z}_C|= \frac{1}{\omega\,C}$
&
$\displaystyle |\tilde{Z}_L|= \omega\,L$
\\\hline\vspacer
$\displaystyle \vec{S}\equiv\frac{1}{\mu_0}\,\vec{E}\times\vec{B}$
&
$\displaystyle \frac{\omega}{k}=c$
&
\\\hline

\end{tabular}
}
\vfill

This exam consists of two problems.  Write only in this booklet.  Be
sure to show your work.

\clearpage

\section*{Problem 1}

Two concentric spherical conducting shells, an inner one of radius $a$
carrying charge $+Q$, and an outer one of radius $b$ carrying charge
$-Q$ form a kind of spherical capacitor.

(a) What would be the capacitance $C$ of this capacitor if the outer
shell is very very close to the inner shell; ie, $b=a+h$, where $h\ll
a$?

\vfill

(b) What is the capacitance $C$ of this capacitor when $b$
\emph{cannot} be approximated as being very close to $a$?  Even if you
know the answer by heart, show how it is derived by explicitly
computing either the voltage difference between the shells, or the
energy density stored in the electric field.

\vfill ~

\clearpage

\section*{Problem 2}

~\hfill\includegraphics{dc_circuit1.eps}\hfill~

What are the steady-state currents through the three resistors (be
sure to specify which is which) and what is the steady-state charge on
the capacitor in the circuit shown?  Show all your work.

\clearpage

[This page intentionally left blank for calculations or other work.]

\end{document}
