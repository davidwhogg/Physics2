\documentclass{article}
\usepackage{graphics}
\begin{document}
\thispagestyle{empty}
\section*{NYU Physics 2 --- Problem Set 10}

\emph{Due by Friday 2002 April 12 at 1pm at Irene Port's office in
Meyer 424.}

\subsection*{Problem 1}

What is the approximate self-inductance $L$ and resistance $R$ of a
gold ring 1~cm in diameter, made of 1~mm diameter wire?  Don't try to
get exact answers, just get an approximate result, correct to
order-of-magnitude.  State your assumptions.  If a pulse of magnetic
field starts a current in the ring, how long does the current take to
decay to zero?

\subsection*{Problem 2}

Compute the self-inductance $L$ of the toroidal solenoid described in
Problem set 9, with $N$ turns of wire, inner radius $a$, outer radius
$b$ and height $h$.  Show your work.  Once you have the general
expression, valuate your answer for $a=4~\mathrm{cm}$,
$b=8~\mathrm{cm}$, $h=4~\mathrm{cm}$, and $N=10^4$.

\subsection*{Problem 3}

\noindent
~\hfill\resizebox{2in}{!}{\includegraphics{dc_circuit2.eps}}\hfill~

The switch in the circuit shown is left open for a very long time.  At
time $t=0$, the switch is closed.  Obtain an equation for the current
$I_L$ in the inductor as a function of time $t$ for $t>0$.  Draw a
graph of $I_L$ against $t$, clearly labeling the minimum and maximum
currents and the exponential timescale.  A very long time later, at
$t=t_1$, the switch is opened again.  Obtain an equation for the
current $I_L$ in the inductor as a function of time $t$ for $t>t_1$.
Draw another graph.  Be sure to show your work and keep your units.

\end{document}
