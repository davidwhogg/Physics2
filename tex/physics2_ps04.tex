\documentclass[12pt]{article}
\usepackage{url, graphicx, epstopdf, amsmath, esint}
\usepackage{physics}

% page layout
\setlength{\topmargin}{-0.25in}
\setlength{\textheight}{9.5in}
\setlength{\headheight}{0in}
\setlength{\headsep}{0in}
\setlength{\parindent}{1.1\baselineskip}
\addtolength{\oddsidemargin}{-0.75in}
\setlength{\marginparwidth}{2in}

% problem formatting
\newcommand{\problemname}{Problem}
\newcounter{problem}
\newcommand{\startproblem}{\paragraph{Problem~\theproblem:}\refstepcounter{problem}}

% words
\newcommand{\foreign}[1]{\textsl{#1}}
\newcommand{\vs}{\foreign{vs}}

% math
\renewcommand{\vec}[1]{\boldsymbol{#1}}
% \newcommand{\dd}{\mathrm{d}} % PROVIDED IN physics PACKAGE
\newcommand{\e}{\mathrm{e}}
% \newcommand{\cross}{\times} % PROVIDED IN physics PACKAGE
% \newcommand{\curl}{\vec{\nabla}\times} % PROVIDED IN physics PACKAGE

% primary units
\newcommand{\rad}{\mathrm{rad}}
\newcommand{\kg}{\mathrm{kg}}
\newcommand{\m}{\mathrm{m}}
\newcommand{\s}{\mathrm{s}}
\newcommand{\A}{\mathrm{A}}

% secondary units
\renewcommand{\deg}{\mathrm{deg}}
\newcommand{\km}{\mathrm{km}}
\newcommand{\cm}{\mathrm{cm}}
\newcommand{\mm}{\mathrm{mm}}
\newcommand{\mum}{\mathrm{\mu m}}
\newcommand{\nm}{\mathrm{nm}}
\newcommand{\ft}{\mathrm{ft}}
\newcommand{\mi}{\mathrm{mi}}
\newcommand{\AU}{\mathrm{AU}}
\newcommand{\ns}{\mathrm{ns}}
\newcommand{\h}{\mathrm{h}}
\newcommand{\yr}{\mathrm{yr}}
\newcommand{\N}{\mathrm{N}}
\newcommand{\J}{\mathrm{J}}
\newcommand{\eV}{\mathrm{eV}}
\newcommand{\MeV}{\mathrm{MeV}}
\newcommand{\W}{\mathrm{W}}
\newcommand{\Pa}{\mathrm{Pa}}
\newcommand{\C}{\mathrm{C}}
\newcommand{\V}{\mathrm{V}}
\newcommand{\ohm}{\mathrm{\Omega}}
\newcommand{\muF}{\mathrm{\mu F}}
\newcommand{\Hz}{\mathrm{Hz}}
\newcommand{\GHz}{\mathrm{GHz}}

% derived units
\newcommand{\mps}{\m\,\s^{-1}}
\newcommand{\mph}{\mi\,\h^{-1}}
\newcommand{\mpss}{\m\,\s^{-2}}
\newcommand{\radps}{\rad\,\s^{-1}}

% random stuff
\sloppy\sloppypar\raggedbottom\frenchspacing\thispagestyle{empty}

\begin{document}

\section*{NYU Physics 2---Problem Set 4}

Due Thursday 2023 February 23 before on Brightspace.
Don't forget to cite your sources and any colloboration you receive.

\startproblem
Look up the total storage capacity of the battery system of a
commercial fully-electric car, like a Tesla or Kia~EV6. It might
be given in amp-hours, or it might be given in joules or
joules-equivalent (like hp-hours or gallons of gasoline
equivalent). And look up also what voltage at which the automobile
power systems run. Is it $12\,\V$ or higher? Now what kind of
capacitance $C$ would it take to replace the battery system with a
(far, far superior) capacitive system? If you started with amp-hours
or coulombs, then also compute what energy is stored when the system
is fully charged. That is, we want to know, in numbers with SI units, the voltage,
charge, capacitance, and stored energy of the system. Now imagine that
the capacitors in the system consist of two very thin sheets separated
by $1\,\mum$.  What would be the total area $A$ you would need for all
those sheets? Again, provide a number in SI units.

\emph{Extra part, not for credit:} Does the energy storage you found
make sense if you want a fully-charged range of hundreds of miles? \emph{How
would you even compute that?}

\emph{Extra part, not for credit:} What are the best dielectrics you can find?
By what factor can you reduce the area you found above by employing an excellent
dielectric?

\startproblem
Expressions for capacitance have units of $\epsilon_0\,R$ where $R$ is a length.
What is the capacitance corresponding to the radius of the Earth?
Interpret this capacitance.
Compare it to the capacitance of capacitors that you can buy (for,
say, less than 100\,USD) on the internet.
Can you buy a capacitor with \emph{way more capacitance than the entire
  Earth}?
If so, how or why? What gives?

\startproblem
Compute the energy stored in a spherical capacitor three ways!
The capacitor will have two concentric spherical plates.
The inner plate has outer radius $a$, and the outer plate has
inner radius $b>a$.
The gap is filled with vacuum.

\textsl{(a)}
Begin by computing the capacitance $C$ of this capacitor, and
use the formula for the energy in a capacitor of capacitance $C$
charged up to charge $Q$.

\textsl{(b)}
Then compute the energy in the capacitor again by looking up a formula
for the energy density stored in the electric field $\vec{E}$, and
integrating over the interior of the capacitor.

\textsl{(c)}
Then compute the energy in the capacitor by using the formula you
computed on the previous Problem Set for a charged sphere, and taking
a difference between two spheres.  Why would that possibly work as a
calculation technique? Energies aren't vectors!

\startproblem
Imagine you add two capacitors in parallel. Make a heuristic argument
that the capacitances should add together. Imagine you add two capacitors
in series. Make a heuristic argument that the inverse-capacitances should
add together. That is, don't use equations, use words and your brain.

\end{document}
