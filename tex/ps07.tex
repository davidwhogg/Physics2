\documentclass{article}
\begin{document}
\thispagestyle{empty}
\section*{NYU Physics 2 --- Problem Set 7}

\emph{Due by Friday 2002 March 22 at 1pm at Irene Port's office in
Meyer 424.}

\subsection*{Problem 1}

When a telegraph cable was first laid across the Atlantic Ocean, the
telegraph signal current traveled one way through the telegraph cable
and returned through the Atlantic Ocean.  Make an estimate of the
total resistance $R$ of the return path through the ocean using the
resistivity of sea water ($\rho= 0.2~\mathrm{\Omega\,m}$) and some
assumptions about the geometry of the Atlantic ocean.  Give your
answer in $\mathrm{\Omega}$.  If the telegraph cable itself was a
single strand of copper ($\rho= 2\times 10^{-8}~\mathrm{\Omega\,m}$)
wire, and the engineers wanted it to contribute only half of the total
resistance of the full circuit, what diameter would they have to have
made the wire?  Your answers to these questions need not be precise,
but state any assumptions or approximations you make.

\subsection*{Problem 2}

A reasonable model of a real battery is that it behaves like a perfect
source of voltage $\cal E$ but with some internal resistance
$R_\mathrm{int}$, connected in series.  A resistor $R$ connected
across this ``real'' battery will dissipate power $P$.  What is the
power dissipation as a function of applied resistance $R$, and at what
resistance $R=R_\mathrm{max}$ is that power dissipation maximized?
Show all your work.

\subsection*{Problem 3}

Sketch the magnetic field $\vec{B}$ lines for a small bar magnet, with
the $N$ pole at one end and the $S$ pole at the other.  Be sure to
mark the directions of the field lines.  The field of the bar magnet
is maintained by (essentially) permanent coherent currents running in
the metal.  Draw on your diagram typical paths of the flowing current.
Be sure to indicate which direction those currents travel (defining
current to be the flow of positive charge).

\end{document}
