\documentclass{article}
\begin{document}
\thispagestyle{empty}
\section*{NYU Physics 2 --- Problem Set 2}

\emph{Due by Friday 2002 February 8 at 1pm at Irene Port's office in
Meyer 424.}

\subsection*{Problem 1}

A dipole of dipole moment $\vec{p}$ and moment of inertia $I$ lies in
a uniform electric field $\vec{E}$.  It is at equilibrium when
$\vec{p}$ is parallel to $\vec{E}$.  If the dipole is displaced (ie,
tilted) by a small angle $\theta$ away from equilibrium and released,
what will be the period of its simple harmonic oscillation?

\subsection*{Problem 2}

Imagine a rod with length $\pi\,R$ and uniformly distributed charge
$Q$ bent into a semi-circle of radius $R$.  What is the direction and
magnitude of the electric field $\vec{E}$ at the center of the
semi-circle?

\subsection*{Problem 3}

A very long (ie, infinite) cylinder of radius $a$ is filled with a
constant positive charge density $\rho$ (charge per unit volume).
Concentric with this cylinder is a thin (ie, zero thickness) outer
uniformly negatively charged cylindrical shell of radius $b$, with
equal and opposite total charge to that contained in the inner
cylinder.  There are no other charges.  Use Gauss's law to determine
the magnitude $E$ of the electric field at any perpendicular distance
$r$ from the center of the cylinders.  Write explicit expressions for
$E$ in the three regions $r<a$, $a<r<b$ and $r>b$.  Make a plot of
$E(r)$, noting the quantitative values of $E$ at $r=0$, $r=a$, $r=b$,
and $r=\infty$.  Are there any discontinuities in $E$?  If so, where?
What is the surface charge density $\sigma$ (charge per unit area) on
the outer cylindrical shell?

\end{document}
