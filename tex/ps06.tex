\documentclass{article}
\begin{document}
\thispagestyle{empty}
\section*{NYU Physics 2 --- Problem Set 6}

\emph{Due by Friday 2002 March 8 at 1pm at Irene Port's office in
Meyer 424.}

\subsection*{Problem 1}

What is the density of conduction electrons in copper wire?  Give your
answer in $\mathrm{cm^{-3}}$.  Look up the density of copper and its
atomic weight, and imagine that there is one conduction electron per
atom (ie, all the other electrons are bound to the nuclei).  A good
car battery can put out about $250~\mathrm{A\,h}$ of total charge at
$120~\mathrm{V}$.  If you simply short-out your car battery with a
copper wire of diameter $5~\mathrm{mm}$, how far does a typical
conduction electron move in the wire?

\subsection*{Problem 2}

Tipler problem 26.90

\subsection*{Problem 3}

A parallel-plate capacitor of plate area $A$ and separation $h$ is
filled with a dielectric of dielectric constant $\kappa$.  Most
dielectrics are not perfect insulators; imagine that the dielectric
has a resistivity $\rho$.  This allows current to slowly ``leak'' from
one plate to another.  At time $t=0$, the positive plate has charge
$Q_0$ and the negative has charge $-Q_0$.  Write an expression for the
charge $Q(t)$ on the positive plate as a function of time $t$.  You
will have to compute the total resistance of the dielectric in terms
of the resistivity and the geometry.

Now do the same for a cylindrical capacitor of length $\ell$, inner
radius $a$ and outer radius $b$.  To get the resistance you will have
to write down an integral.  (\emph{Hint:} You can treat
infinitesimally thin cylindrical shells of the dielectric as
infinitesimal resistors in series.)

\end{document}
