\documentclass{article}
\begin{document}
\thispagestyle{empty}
\section*{NYU Physics 2 --- Problem Set 5}

\emph{Due by Friday 2002 March 1 at 1pm at Irene Port's office in
Meyer 424.}

\subsection*{Problem 1}

Imagine that you have a glass of water in which all of the water
molecules are aligned, so that their positive ends all point upwards.
Imagine that each molecule can be treated as a simple dipole
consisting of two charges $q$ and $-q$ separated by a small vertical
distance $a$.  In this case, the water is like two superimposed charge
densities, one positive and one negative, displaced vertically by a
tiny distance $a$.  If the water in a glass is in this aligned state,
what is the magnitude of the electric field $E$ in the water?  Your
answer should depend on the product $q\,a$, which is the dipole moment
of the water molecule, which you can look up in your book.  What would
be the potential difference $V$ in volts from the top to the bottom of
a glass of water in this aligned state?  Do you think this ever
happens in nature?  Clearly state all your assumptions, approximations
and estimates.

\subsection*{Problem 2}

Consider a parallel-plate capacitor of width $X$, depth $Y$ and plate
separation $h$, charged to a total charge $Q$, filled with a slab of
dielectric of constant $\kappa$, as discussed in lecture.  If you pull
the dielectric slab out of the capacitor by a distance $x$, you change
the capacitance, as shown in lecture.  Compute the capacitance $C$ and
total energy $U$ in the capacitor as a function of $x$.  Compute the
force $F$ required as a function of $x$ to remove the slab.

\subsection*{Problem 3}

Consider a parallel-plate capacitor of the same dimensions and charge
as in Problem~2, but now filled with vacuum, not a dielectric.  If you
pull apart the plates of the capacitor (ie, increase the separation
$h$), you increase the volume of the capacitor.  Compute the integral
of the energy density in the electric field to get the total energy
$U$ in the capacitor as a function of $h$.  What force $F$ is required
to pull apart the plates?  Is this the answer you would have expected
from, say, $F=q\,E$?

\end{document}
