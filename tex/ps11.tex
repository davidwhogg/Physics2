\documentclass{article}
\usepackage{graphics}
\begin{document}
\thispagestyle{empty}
\section*{NYU Physics 2 --- Problem Set 11}

\emph{Due by Friday 2002 April 19 at 1pm at Irene Port's office in
Meyer 424.}

\subsection*{Problem 1}

Compute the resonant frequency of a $\ell=10~\mathrm{m}$ length of
coaxial cable (like a television cable), shorted-out at both ends.
Estimate the capacitance as you would for a long, thin cylindrical
capacitor of length $\ell$, inner radius $a=0.5~\mathrm{mm}$ and outer
radius $b=2~\mathrm{mm}$.  It is filled with a plastic dielectric, but
assume that the dielectric constant $\kappa=1$ (bad assumption,
perhaps).  Estimate the inductance as you would for a single loop of
length $\ell$ and width $(b-a)$.  We will accept approximate answers,
but if you think about the geometry, you can actually solve this
problem exactly.  How does the fundamental frequency depend on the
length $\ell$?  Does it increase, decrease or stay the same?

\subsection*{Problem 2}

\noindent
~\hfill\includegraphics{dc_circuit4.eps}\hfill~

In the circuit shown, the inductance is $L=1~\mathrm{mH}$, the
capacitance is $C= 2~\mathrm{\mu F}$, and the resistor is
$R=0.01~\mathrm{\Omega}$.  The capacitor is initially charged with
$1~\mathrm{nC}$ and then the switch is closed at time $t=0$.  Compare
the resistance timescale $R/L$ to the natural frequency
$1/\sqrt{L\,C}$.  Will the circuit oscillate for many oscillations
before the current dies away, or will it damp out quickly?  Roughly
how many oscillations will take place before the amplitude of
oscillations has dropped to $1/e$ of its original value?

\subsection*{Problem 3}

\noindent
~\hfill\includegraphics{dc_circuit3.eps}\hfill~

What is the magnitude of the total impedance of the $R$, $L$, $C$
combination in the circuit shown?  What is the time-average power
dissipation $\left<P_R\right>$ in the resistor when
$\omega=(L\,C)^{-1/2}$?  What is the time-average power dissipation in
the resistor when $\omega=2\,(L\,C)^{-1/2}$?

\end{document}
