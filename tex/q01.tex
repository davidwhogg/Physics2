\documentclass[12pt]{article}
\usepackage{graphics}
\begin{document}

\section*{NYU Physics 2---In-class Exam 1}

\vfill

\paragraph{Name:} ~
\paragraph{email:} ~

\vfill
\noindent
\resizebox{\textwidth}{!}{\newcommand{\vspacer}{\rule[-2.75ex]{0pt}{7ex}}%
\newcommand{\iint}{\makebox[0.5ex][l]{$\displaystyle\int$}\int}%
\newcommand{\oiint}{\makebox[-0.20ex][l]{$\bigcirc$}\iint}%
\begin{tabular}{|c|c|c|}%
\hline\vspacer
$\displaystyle \vec{F}= q\,(\vec{E}+\vec{v}\times\vec{B})$
&
$\displaystyle
  \oiint \vec{E}\cdot\hat{n}\,\mathrm{d}A= \frac{Q_\mathrm{encl}}{\epsilon_0}$
&
$\displaystyle
  \vec{F}_{12}= \frac{k\,q_1\,q_2}{|\vec{r}_1-\vec{r}_2|^3}
    \,(\vec{r}_1-\vec{r}_2)$
\\\hline\vspacer
$\displaystyle \frac{1}{4\pi\,\epsilon_0}= k=
   9\times 10^9~\mathrm{\frac{N\,m^2}{C^2}}$
&
$\displaystyle V_{ab}= -\int_a^b \vec{E}\cdot\mathrm{d}\vec{r}= \phi_b-\phi_a$
&
$\displaystyle W= q\,V$
\\\hline\vspacer
$\displaystyle \phi(\vec{r})= \frac{k\,q}{|\vec{r}|}$
&
$\displaystyle C\equiv \frac{Q}{V}$
&
$\displaystyle U= \frac{1}{2}\,C\,V^2$
\\\hline\vspacer
$\displaystyle
   u= \frac{\epsilon_0}{2}\,|\vec{E}|^2+\frac{1}{2\,\mu_0}\,|\vec{B}|^2$
&
$\displaystyle \vec{E}_\mathrm{inside}\approx\frac{1}{\kappa}\,\vec{E}_\mathrm{vacuum}$
&
$\displaystyle R= \frac{\rho\,\ell}{A}$
\\\hline\vspacer
$\displaystyle V= I\,R$
&
$\displaystyle P= I\,V$
&
$\displaystyle \mathrm{d}\vec{F}= I\,\mathrm{d}\vec{\ell}\times\vec{B}$
\\\hline\vspacer
$\displaystyle \mathrm{d}\vec{B}= \frac{\mu_0}{4\pi}\,\frac{I\,\mathrm{d}\vec{\ell}\times\hat{r}}{r^2}$
&
$\displaystyle \oint \vec{B}\cdot\mathrm{d}\vec{r}= \mu_0\,I
   +\mu_0\,\epsilon_0\,\frac{\mathrm{d}}{\mathrm{d}t}
   \iint\vec{E}\cdot\hat{n}\,\mathrm{d}A$
&
$\displaystyle \mu_0= 4\pi\times 10^{-7}~\mathrm{T\,m\,A^{-1}}$
\\\hline\vspacer
$\displaystyle
  \oiint\vec{B}\cdot\hat{n}\,\mathrm{d}A= 0$
&
$\displaystyle \oint \vec{E}\cdot\mathrm{d}\vec{r}=
   -\frac{\mathrm{d}}{\mathrm{d}t}
   \iint\vec{B}\cdot\hat{n}\,\mathrm{d}A$
&
$\displaystyle V= -L\,\frac{\mathrm{d}I}{\mathrm{d}t}$
\\\hline\vspacer
$\displaystyle U= \frac{1}{2}\,L\,I^2$
&
$\displaystyle |\tilde{Z}_C|= \frac{1}{\omega\,C}$
&
$\displaystyle |\tilde{Z}_L|= \omega\,L$
\\\hline\vspacer
$\displaystyle \vec{S}\equiv\frac{1}{\mu_0}\,\vec{E}\times\vec{B}$
&
$\displaystyle \frac{\omega}{k}=c$
&
\\\hline

\end{tabular}
}
\vfill

This exam consists of two problems.  Write only in this booklet.  Be
sure to show your work.

\clearpage

\section*{Problem 1}

Consider a thin disk of radius $2~\mathrm{cm}$ in the $x$--$y$ plane,
centered on the origin, uniformly coated with a charge of
$1~\mathrm{nC}$.

(a) What is the force (magnitude and direction) on a test charge
$q=0.1~\mathrm{nC}$ on the $z$ axis at $z=0.01~\mathrm{mm}$?  Draw a
simple diagram of the situation.  There is no need to give an exact
answer, but you must explain and justify any approximations you make.

\vfill

(b) What is the force (magnitude and direction) on a test charge
$q=0.1~\mathrm{nC}$ on the $z$ axis at $z=10~\mathrm{m}$?  Again,
explain and justify any approximations you make.

\vfill ~

\clearpage

\section*{Problem 2}

A non-conducting sphere of radius $5~\mathrm{cm}$ centered on the
origin contains a uniformly distributed charge of $1~\mathrm{C}$.  A
non-conducting sphere of radius $10~\mathrm{cm}$ has its center a
distance $10~\mathrm{m}$ away and contains a uniformly distributed
charge of $-1~\mathrm{C}$.

(a) Which sphere is at higher potential?  Explain your answer.

\vfill

(b) What is the potential difference between the spheres, in $V$?
Clearly state any approximations you make in getting your answer.

\vfill ~

\clearpage

[This page intentionally left blank for calculations or other work.]

\end{document}
