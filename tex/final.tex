\documentclass[12pt]{article}
\usepackage{graphics}
\begin{document}
\thispagestyle{empty}
\setcounter{page}{0}
\section*{NYU Physics 2---Final Exam}

\vfill

\paragraph{Name:} ~
\paragraph{email:} ~

\vfill
\noindent
\resizebox{\textwidth}{!}{\newcommand{\vspacer}{\rule[-2.75ex]{0pt}{7ex}}%
\newcommand{\iint}{\makebox[0.5ex][l]{$\displaystyle\int$}\int}%
\newcommand{\oiint}{\makebox[-0.20ex][l]{$\bigcirc$}\iint}%
\begin{tabular}{|c|c|c|}%
\hline\vspacer
$\displaystyle \vec{F}= q\,(\vec{E}+\vec{v}\times\vec{B})$
&
$\displaystyle
  \oiint \vec{E}\cdot\hat{n}\,\mathrm{d}A= \frac{Q_\mathrm{encl}}{\epsilon_0}$
&
$\displaystyle
  \vec{F}_{12}= \frac{k\,q_1\,q_2}{|\vec{r}_1-\vec{r}_2|^3}
    \,(\vec{r}_1-\vec{r}_2)$
\\\hline\vspacer
$\displaystyle \frac{1}{4\pi\,\epsilon_0}= k=
   9\times 10^9~\mathrm{\frac{N\,m^2}{C^2}}$
&
$\displaystyle V_{ab}= -\int_a^b \vec{E}\cdot\mathrm{d}\vec{r}= \phi_b-\phi_a$
&
$\displaystyle W= q\,V$
\\\hline\vspacer
$\displaystyle \phi(\vec{r})= \frac{k\,q}{|\vec{r}|}$
&
$\displaystyle C\equiv \frac{Q}{V}$
&
$\displaystyle U= \frac{1}{2}\,C\,V^2$
\\\hline\vspacer
$\displaystyle
   u= \frac{\epsilon_0}{2}\,|\vec{E}|^2+\frac{1}{2\,\mu_0}\,|\vec{B}|^2$
&
$\displaystyle \vec{E}_\mathrm{inside}\approx\frac{1}{\kappa}\,\vec{E}_\mathrm{vacuum}$
&
$\displaystyle R= \frac{\rho\,\ell}{A}$
\\\hline\vspacer
$\displaystyle V= I\,R$
&
$\displaystyle P= I\,V$
&
$\displaystyle \mathrm{d}\vec{F}= I\,\mathrm{d}\vec{\ell}\times\vec{B}$
\\\hline\vspacer
$\displaystyle \mathrm{d}\vec{B}= \frac{\mu_0}{4\pi}\,\frac{I\,\mathrm{d}\vec{\ell}\times\hat{r}}{r^2}$
&
$\displaystyle \oint \vec{B}\cdot\mathrm{d}\vec{r}= \mu_0\,I
   +\mu_0\,\epsilon_0\,\frac{\mathrm{d}}{\mathrm{d}t}
   \iint\vec{E}\cdot\hat{n}\,\mathrm{d}A$
&
$\displaystyle \mu_0= 4\pi\times 10^{-7}~\mathrm{T\,m\,A^{-1}}$
\\\hline\vspacer
$\displaystyle
  \oiint\vec{B}\cdot\hat{n}\,\mathrm{d}A= 0$
&
$\displaystyle \oint \vec{E}\cdot\mathrm{d}\vec{r}=
   -\frac{\mathrm{d}}{\mathrm{d}t}
   \iint\vec{B}\cdot\hat{n}\,\mathrm{d}A$
&
$\displaystyle V= -L\,\frac{\mathrm{d}I}{\mathrm{d}t}$
\\\hline\vspacer
$\displaystyle U= \frac{1}{2}\,L\,I^2$
&
$\displaystyle |\tilde{Z}_C|= \frac{1}{\omega\,C}$
&
$\displaystyle |\tilde{Z}_L|= \omega\,L$
\\\hline\vspacer
$\displaystyle \vec{S}\equiv\frac{1}{\mu_0}\,\vec{E}\times\vec{B}$
&
$\displaystyle \frac{\omega}{k}=c$
&
\\\hline

\end{tabular}
}
\vfill

This exam consists of six problems.  Write only in this booklet.  Be
sure to show your work.

\clearpage

\section*{Problem 1}

An oil drop has a mass of $7\times 10^{-14}~\mathrm{kg}$ and a net
charge of $4.8\times 10^{-19}~\mathrm{C}$.  An upward electric force
just balances the downward force of gravity so that the oil drop is
stationary.  What is the direction and magnitude of the electric
field?

\clearpage

\section*{Problem 2}

Draw electric field lines and equipotentials in the $x$--$y$ plane for
a thin, positively charged, conducting, equilateral triangle lying in
the plane.  Draw your picture carefully and annotate it so that it
clearly demonstrates your knowledge.

\vfill

\noindent~\hfill\includegraphics{final1.eps}\hfill~

\clearpage

\section*{Problem 3}

Consider a cylindrical $100~\mathrm{\Omega}$ resistor of radius
$1~\mathrm{mm}$ and length $5~\mathrm{mm}$.

(a) If the resistor is carrying a current of $0.1~\mathrm{A}$, what
is the magnitude and direction of the electric field inside?  Draw a
simple diagram.

\vfill

(b) Since there is an electric field in the resistor, the resistor has
a small capacitance.  Make a (very) approximate estimate of the
capacitance $C$ of the resistor and compute the AC frequency $\omega$
at which the impedance due to the resistor's capacitance exceeds the
impedance due to its resistance.

\vfill ~

\clearpage

\section*{Problem 4}

Sketch the magnetic field lines in the $x$--$y$ plane for two parallel
wires carrying equal and opposite currents in the $z$ direction.  Draw
your diagram carefully so that it clearly demonstrates your knowledge.

\vfill

\noindent~\hfill\includegraphics{final2.eps}\hfill~

\clearpage

\section*{Problem 5}

A metal disk of radius $a$ rotates at angular frequency $\omega$
(radians per second) inside a uniform magnetic field of magnitude $B$
perpendicular to the plane of the disk.

(a) What is the force $\vec{F}_B$ (magnitude and direction) due to the
magnetic field on a point charge $+Q$ at point $P$ inside the rotating
disk, a distance $r$ from the center of the disk?

~

\noindent\includegraphics{final3.eps}

~

(b) What is the potential difference $V$ between the center of the
disk and the outer edge?

\vfill ~

\clearpage

\section*{Problem 6}

\noindent
~\hfill\includegraphics{dc_circuit5.eps}\hfill~\\
In the circuit shown, the switch is closed at time $t=0$.

(a) What are the currents $I_1$ and $I_2$ in resistors $R_1$ and
$R_2$, immediately after the switch is closed?  Clearly justify your
answer with equations or an explanation.

\vfill

(b) What are the currents $I_1$ and $I_2$ as $t\rightarrow\infty$?
Note that you do \emph{not} have to get part (a) to get this part, but
explain your answer.

\vfill ~

\clearpage
\thispagestyle{empty}
[This page intentionally left blank for calculations or other work.]

\end{document}
