\documentclass[12pt]{article}
\usepackage{url, graphicx, epstopdf, amsmath, esint}
\usepackage{physics}

% page layout
\setlength{\topmargin}{-0.25in}
\setlength{\textheight}{9.5in}
\setlength{\headheight}{0in}
\setlength{\headsep}{0in}
\setlength{\parindent}{1.1\baselineskip}
\addtolength{\oddsidemargin}{-0.75in}
\setlength{\marginparwidth}{2in}

% problem formatting
\newcommand{\problemname}{Problem}
\newcounter{problem}
\newcommand{\startproblem}{\paragraph{Problem~\theproblem:}\refstepcounter{problem}}

% words
\newcommand{\foreign}[1]{\textsl{#1}}
\newcommand{\vs}{\foreign{vs}}

% math
\renewcommand{\vec}[1]{\boldsymbol{#1}}
% \newcommand{\dd}{\mathrm{d}} % PROVIDED IN physics PACKAGE
\newcommand{\e}{\mathrm{e}}
% \newcommand{\cross}{\times} % PROVIDED IN physics PACKAGE
% \newcommand{\curl}{\vec{\nabla}\times} % PROVIDED IN physics PACKAGE

% primary units
\newcommand{\rad}{\mathrm{rad}}
\newcommand{\kg}{\mathrm{kg}}
\newcommand{\m}{\mathrm{m}}
\newcommand{\s}{\mathrm{s}}
\newcommand{\A}{\mathrm{A}}

% secondary units
\renewcommand{\deg}{\mathrm{deg}}
\newcommand{\km}{\mathrm{km}}
\newcommand{\cm}{\mathrm{cm}}
\newcommand{\mm}{\mathrm{mm}}
\newcommand{\mum}{\mathrm{\mu m}}
\newcommand{\nm}{\mathrm{nm}}
\newcommand{\ft}{\mathrm{ft}}
\newcommand{\mi}{\mathrm{mi}}
\newcommand{\AU}{\mathrm{AU}}
\newcommand{\ns}{\mathrm{ns}}
\newcommand{\h}{\mathrm{h}}
\newcommand{\yr}{\mathrm{yr}}
\newcommand{\N}{\mathrm{N}}
\newcommand{\J}{\mathrm{J}}
\newcommand{\eV}{\mathrm{eV}}
\newcommand{\MeV}{\mathrm{MeV}}
\newcommand{\W}{\mathrm{W}}
\newcommand{\Pa}{\mathrm{Pa}}
\newcommand{\C}{\mathrm{C}}
\newcommand{\V}{\mathrm{V}}
\newcommand{\ohm}{\mathrm{\Omega}}
\newcommand{\muF}{\mathrm{\mu F}}
\newcommand{\Hz}{\mathrm{Hz}}
\newcommand{\GHz}{\mathrm{GHz}}

% derived units
\newcommand{\mps}{\m\,\s^{-1}}
\newcommand{\mph}{\mi\,\h^{-1}}
\newcommand{\mpss}{\m\,\s^{-2}}
\newcommand{\radps}{\rad\,\s^{-1}}

% random stuff
\sloppy\sloppypar\raggedbottom\frenchspacing\thispagestyle{empty}

\begin{document}

\section*{NYU Physics 2---Problem Set 5}

Due Thursday 2020 March 05 at the beginning of lecture.

\paragraph{Problem~\theproblem:}\refstepcounter{problem}%
Look up the dielectric constants for materials.
In standard, inexpensive, commercial capacitors, what kinds of dielectrics are used?
How much lower would the capacitances be for those capacitors if the gaps were air gaps instead
of dielectric-filled gaps?
Are there materials with dielectric constants in the thousands? If so, what are their properties?

\paragraph{Problem~\theproblem:}\refstepcounter{problem}%
Imagine that you have a glass of water in which all of the water
molecules are aligned, so that their positive ends all point upwards.
Imagine that each molecule can be treated as a simple dipole
consisting of two charges $q$ and $-q$ separated by a small vertical
distance $a$.  In this case, the water is like two superimposed charge
densities, one positive and one negative, displaced vertically by a
tiny distance $a$.  If the water in a glass is in this aligned state,
what is the magnitude of the electric field $E$ in the water?  Your
answer should depend on the product $q\,a$, which is the dipole moment
of the water molecule, which you can look up (be careful with units!).  What would
be the potential difference $V$ in volts from the top to the bottom of
a pint glass of water in this aligned state? Do you think this ever
happens in nature? Clearly state all your assumptions, approximations
and estimates.

\paragraph{Problem~\theproblem:}\refstepcounter{problem}%
A circuit consists of a capacitor $C$ and a resistor $R$,
and nothing else.
At time $t=0$, the capacitor contains charge $q_0$.
What is the instantaneous flow of current $I_0$ in the circuit at time $t=0$?
As time goes on, charge will flow and the capacitor will discharge.
Write down the equation relating $q(t)$ and its time derivative $\dd q/\dd t$ (which
is related to the current) and show that this equation is satisfied if the charge
as a function of time is given by
\begin{equation}
q(t) = q_0\,\exp \alpha\,t
\end{equation}
where $\alpha$ has some expression in terms of $R$ and $C$. What is that expression?

\paragraph{Problem~\theproblem:}\refstepcounter{problem}%
Consider a parallel-plate capacitor of width $X$, depth $Y$ and plate
separation $h$, charged to a total charge $Q$, filled with a slab of
dielectric of constant $\kappa$, as discussed in lecture.  If you pull
the dielectric slab out of the capacitor by a distance $x$, you change
the capacitance. Compute the capacitance $C$ and
total energy $U$ in the capacitor as a function of $x$. You will have to
ignore edge effects---that is, assume that the separation is very small
relative to all other dimensions.

\textsl{Bonus part, not for credit:} Compute the
force $F$ required as a function of $x$ to remove the slab.

\textsl{Bonus part, not for credit:} How do your answers change if you
hold the voltage $V$ on the capacitor fixed instead of the charge $Q$?

\end{document}
