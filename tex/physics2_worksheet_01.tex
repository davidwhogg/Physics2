\documentclass[12pt]{article}
\usepackage{url, graphicx, epstopdf, amsmath, esint}

% page layout
\setlength{\topmargin}{-0.25in}
\setlength{\textheight}{9.5in}
\setlength{\headheight}{0in}
\setlength{\headsep}{0in}
\setlength{\parindent}{1.1\baselineskip}
\addtolength{\oddsidemargin}{-0.75in}
\setlength{\marginparwidth}{2in}

% problem formatting
\newcommand{\problemname}{Problem}
\newcounter{problem}
\newcommand{\startproblem}{\paragraph{Problem~\theproblem:}\refstepcounter{problem}}

% words
\newcommand{\foreign}[1]{\textsl{#1}}
\newcommand{\vs}{\foreign{vs}}

% math
\renewcommand{\vec}[1]{\boldsymbol{#1}}
\newcommand{\dd}{\mathrm{d}}
\newcommand{\e}{\mathrm{e}}
\newcommand{\cross}{\times}
\newcommand{\curl}{\vec{\nabla}\times}

% primary units
\newcommand{\rad}{\mathrm{rad}}
\newcommand{\kg}{\mathrm{kg}}
\newcommand{\m}{\mathrm{m}}
\newcommand{\s}{\mathrm{s}}
\newcommand{\A}{\mathrm{A}}

% secondary units
\renewcommand{\deg}{\mathrm{deg}}
\newcommand{\km}{\mathrm{km}}
\newcommand{\cm}{\mathrm{cm}}
\newcommand{\mm}{\mathrm{mm}}
\newcommand{\mum}{\mathrm{\mu m}}
\newcommand{\nm}{\mathrm{nm}}
\newcommand{\ft}{\mathrm{ft}}
\newcommand{\mi}{\mathrm{mi}}
\newcommand{\AU}{\mathrm{AU}}
\newcommand{\ns}{\mathrm{ns}}
\newcommand{\h}{\mathrm{h}}
\newcommand{\yr}{\mathrm{yr}}
\newcommand{\N}{\mathrm{N}}
\newcommand{\J}{\mathrm{J}}
\newcommand{\eV}{\mathrm{eV}}
\newcommand{\MeV}{\mathrm{MeV}}
\newcommand{\W}{\mathrm{W}}
\newcommand{\Pa}{\mathrm{Pa}}
\newcommand{\C}{\mathrm{C}}
\newcommand{\V}{\mathrm{V}}
\newcommand{\ohm}{\mathrm{\Omega}}
\newcommand{\muF}{\mathrm{\mu F}}
\newcommand{\Hz}{\mathrm{Hz}}
\newcommand{\GHz}{\mathrm{GHz}}

% derived units
\newcommand{\mps}{\m\,\s^{-1}}
\newcommand{\mph}{\mi\,\h^{-1}}
\newcommand{\mpss}{\m\,\s^{-2}}
\newcommand{\radps}{\rad\,\s^{-1}}

% random stuff
\sloppy\sloppypar\raggedbottom\frenchspacing\thispagestyle{empty}

\begin{document}

\section*{NYU Physics 2---Worksheet 01}

Try making 2-dimensional sketches of the electric field for some simple configurations.

\startproblem%
Draw two charges, of charge $+Q$ and $-Q$, separated by a tiny separation $a$, at the center of your paper.
This is an electric dipole and we will talk much more about it soon, but it is fun to work out the field direction everywhere.
Take the vector additions and subtractions of the fields to determine the direction of the field at many points around your paper.
Consult with your neighbors to see if you agree on things.
What will happen to the field if you go very far away from this dipole?

\startproblem%
Now draw four charges, in a tiny square of side length $a$, with the $+Q$s on two diagonally opposite places, and the $-Q$s in the other two spots.
That is, draw a pair of anti-aligned dipoles.
This is an electric quadrupole.
Again, use vector addition and also symmetry arguments to argue for the direction of the field everywhere in the plane.

\paragraph{Discussion:}
If you take the limit as $a\rightarrow 0$ but holding the product $Q\,a$ fixed, do you see why the first problem looks like ``taking the derivative'' of the field of a point charge?
What do you think the second problem becomes as you take a similar limit?

\end{document}
