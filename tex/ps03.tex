\documentclass{article}
\begin{document}
\thispagestyle{empty}
\section*{NYU Physics 2 --- Problem Set 3}

\emph{Due by Friday 2002 February 15 at 1pm at Irene Port's office in
Meyer 424.}

\subsection*{Problem 1}

A cylindrical, high-tension wire of radius $2~\mathrm{cm}$ is held at
an electrostatic potential difference of $10^4~\mathrm{V}$ relative to
the ground.  It is suspended at a height of $50~\mathrm{m}$ from the
ground.  What is the magnitude $E$ of the electric field right at the
surface of the wire?  How small can the radius of the wire be made
before the air breaks down around it?  State any assumptions you make
in getting your answers.

\subsection*{Problem 2}

A thin, flat plane with surface charge density $\sigma$ is inserted
between two large, flat, conducting, parallel plates.  Both the plane
and the plates are normal to the $x$ direction, with the charged plane
defined to be at $x=0$.  One conducting plate is at $x= -x_1$ and the
other is at $x=x_2$.  What is the electric field everywhere between
the two plates if the two plates are connected by a conducting wire,
as shown in lecture?  You will have to use Gauss's law, the symmetry
of the problem, and the fact that the surface of a conductor is an
equipotential.  Assume that the plates and plane are all large enough
that they can be treated as ``infinite'' and the edges can be ignored.

\subsection*{Problem 3}

Consider a thin, square, charged plate of metal in the $x$--$y$ plane.
Sketch the equipotentials and field lines in the $x$--$y$ plane,
noting explicitly on your diagram whatever you can about the shape and
density of field lines and equipotential lines.

\end{document}
