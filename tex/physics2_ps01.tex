\documentclass[12pt]{article}
\usepackage{url, graphicx, epstopdf, amsmath, esint}
\usepackage{physics}

% page layout
\setlength{\topmargin}{-0.25in}
\setlength{\textheight}{9.5in}
\setlength{\headheight}{0in}
\setlength{\headsep}{0in}
\setlength{\parindent}{1.1\baselineskip}
\addtolength{\oddsidemargin}{-0.75in}
\setlength{\marginparwidth}{2in}

% problem formatting
\newcommand{\problemname}{Problem}
\newcounter{problem}
\newcommand{\startproblem}{\paragraph{Problem~\theproblem:}\refstepcounter{problem}}

% words
\newcommand{\foreign}[1]{\textsl{#1}}
\newcommand{\vs}{\foreign{vs}}

% math
\renewcommand{\vec}[1]{\boldsymbol{#1}}
% \newcommand{\dd}{\mathrm{d}} % PROVIDED IN physics PACKAGE
\newcommand{\e}{\mathrm{e}}
% \newcommand{\cross}{\times} % PROVIDED IN physics PACKAGE
% \newcommand{\curl}{\vec{\nabla}\times} % PROVIDED IN physics PACKAGE

% primary units
\newcommand{\rad}{\mathrm{rad}}
\newcommand{\kg}{\mathrm{kg}}
\newcommand{\m}{\mathrm{m}}
\newcommand{\s}{\mathrm{s}}
\newcommand{\A}{\mathrm{A}}

% secondary units
\renewcommand{\deg}{\mathrm{deg}}
\newcommand{\km}{\mathrm{km}}
\newcommand{\cm}{\mathrm{cm}}
\newcommand{\mm}{\mathrm{mm}}
\newcommand{\mum}{\mathrm{\mu m}}
\newcommand{\nm}{\mathrm{nm}}
\newcommand{\ft}{\mathrm{ft}}
\newcommand{\mi}{\mathrm{mi}}
\newcommand{\AU}{\mathrm{AU}}
\newcommand{\ns}{\mathrm{ns}}
\newcommand{\h}{\mathrm{h}}
\newcommand{\yr}{\mathrm{yr}}
\newcommand{\N}{\mathrm{N}}
\newcommand{\J}{\mathrm{J}}
\newcommand{\eV}{\mathrm{eV}}
\newcommand{\MeV}{\mathrm{MeV}}
\newcommand{\W}{\mathrm{W}}
\newcommand{\Pa}{\mathrm{Pa}}
\newcommand{\C}{\mathrm{C}}
\newcommand{\V}{\mathrm{V}}
\newcommand{\ohm}{\mathrm{\Omega}}
\newcommand{\muF}{\mathrm{\mu F}}
\newcommand{\Hz}{\mathrm{Hz}}
\newcommand{\GHz}{\mathrm{GHz}}

% derived units
\newcommand{\mps}{\m\,\s^{-1}}
\newcommand{\mph}{\mi\,\h^{-1}}
\newcommand{\mpss}{\m\,\s^{-2}}
\newcommand{\radps}{\rad\,\s^{-1}}

% random stuff
\sloppy\sloppypar\raggedbottom\frenchspacing\thispagestyle{empty}

\begin{document}

\section*{NYU Physics 2---Problem Set 1}

Due Thursday 2023 February 02 on Brightspace.
Submit your Problem Set in a generally accessible file format, such as JPEG, PNG, or PDF
(no proprietary formats, such as HEIC, please).
You are encouraged to work together.
Don't forget to \emph{explicitly give credit} to any resources or people
you consulted in completing these these problems.

\startproblem%
Coulomb's law can be summarized as
\begin{equation}
  |\vec{F}| = \frac{1}{4\pi\,\epsilon_0}\,\frac{Q^2}{R^2}~,
\end{equation}
where $\vec{F}$ is a force, $Q$ is a charge, and $R$ is a length.
Use the \emph{units} or \emph{dimensions} of this problem to write
down dimensionally correct expressions for
\textsl{(a)}~an energy,
\textsl{(b)}~a voltage (energy per charge),
\textsl{(c)}~an electric field (voltage per meter).

You may only use the inputs $\epsilon_0$, $Q$, and $R$ in your answers, and all
we care about are the units, not the detailed form (or $4\pi$ factors).
That is, we are trying to get a dimensionally correct answer, not a correct answer.

\startproblem%
Imagine a rod with length $\pi\,R$ and uniformly distributed charge
$Q$ bent into a semi-circle of radius $R$.  What is the direction and
magnitude of the electric field $\vec{E}$ produced by this object
at the point that is precisely at the center of the semi-circle?
Show your work.

\startproblem%
Imagine you have a very thin plastic hoop of diameter $D$, which is
charged up to total net electrostatic charge $Q$.
The hoop is thin in the sense that the plastic rod of which it is
made has a diameter $d\ll D$.
Computing the electric field everywhere from this hoop is hard!
But there are three places that are easy.
Compute the electric field (possibly in a limit, or approximately)
at these three locations, and give a clear argument for your answer
in each case:

\textsl{(a)} exactly at the center of the hoop;

\textsl{(b)} at a position that is a very small perpendicular
distance $r_\perp$ from some point on edge of the hoop,
such that $d\ll r\ll D$; and

\textsl{(c)} at very large distance $R$, such that $R\gg D$.

In each case, give the magnitude and direction of the field. You might
need to use words (rather than formulae) to express the direction
unambiguously.
Assume that there are no other charges anywhere!

\startproblem%
Imagine that the charge on the proton is bigger in magnitude than the
charge on the electron by 1 part in a trillion ($10^{12}$).  What is
the magnitude of the electrostatic force between you and a friend who
is $1~\mathrm{m}$ away?  How much total work did the two of you have
to do to get that close to one another (assuming you started very far
apart)? You might need to look up electrostatic energy to solve that last part.
And you might need to make the spherical-student approximation!

The idea here is to get an order-of-magnitude estimate, not a precise answer.

\end{document}
