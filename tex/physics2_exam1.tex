\documentclass[12pt]{article}
\usepackage{url, graphicx, epstopdf, amsmath, esint}
\usepackage{physics}

% page layout
\setlength{\topmargin}{-0.25in}
\setlength{\textheight}{9.5in}
\setlength{\headheight}{0in}
\setlength{\headsep}{0in}
\setlength{\parindent}{1.1\baselineskip}
\addtolength{\oddsidemargin}{-0.75in}
\setlength{\marginparwidth}{2in}

% problem formatting
\newcommand{\problemname}{Problem}
\newcounter{problem}
\newcommand{\startproblem}{\paragraph{Problem~\theproblem:}\refstepcounter{problem}}

% words
\newcommand{\foreign}[1]{\textsl{#1}}
\newcommand{\vs}{\foreign{vs}}

% math
\renewcommand{\vec}[1]{\boldsymbol{#1}}
% \newcommand{\dd}{\mathrm{d}} % PROVIDED IN physics PACKAGE
\newcommand{\e}{\mathrm{e}}
% \newcommand{\cross}{\times} % PROVIDED IN physics PACKAGE
% \newcommand{\curl}{\vec{\nabla}\times} % PROVIDED IN physics PACKAGE

% primary units
\newcommand{\rad}{\mathrm{rad}}
\newcommand{\kg}{\mathrm{kg}}
\newcommand{\m}{\mathrm{m}}
\newcommand{\s}{\mathrm{s}}
\newcommand{\A}{\mathrm{A}}

% secondary units
\renewcommand{\deg}{\mathrm{deg}}
\newcommand{\km}{\mathrm{km}}
\newcommand{\cm}{\mathrm{cm}}
\newcommand{\mm}{\mathrm{mm}}
\newcommand{\mum}{\mathrm{\mu m}}
\newcommand{\nm}{\mathrm{nm}}
\newcommand{\ft}{\mathrm{ft}}
\newcommand{\mi}{\mathrm{mi}}
\newcommand{\AU}{\mathrm{AU}}
\newcommand{\ns}{\mathrm{ns}}
\newcommand{\h}{\mathrm{h}}
\newcommand{\yr}{\mathrm{yr}}
\newcommand{\N}{\mathrm{N}}
\newcommand{\J}{\mathrm{J}}
\newcommand{\eV}{\mathrm{eV}}
\newcommand{\MeV}{\mathrm{MeV}}
\newcommand{\W}{\mathrm{W}}
\newcommand{\Pa}{\mathrm{Pa}}
\newcommand{\C}{\mathrm{C}}
\newcommand{\V}{\mathrm{V}}
\newcommand{\ohm}{\mathrm{\Omega}}
\newcommand{\muF}{\mathrm{\mu F}}
\newcommand{\Hz}{\mathrm{Hz}}
\newcommand{\GHz}{\mathrm{GHz}}

% derived units
\newcommand{\mps}{\m\,\s^{-1}}
\newcommand{\mph}{\mi\,\h^{-1}}
\newcommand{\mpss}{\m\,\s^{-2}}
\newcommand{\radps}{\rad\,\s^{-1}}

% random stuff
\sloppy\sloppypar\raggedbottom\frenchspacing\thispagestyle{empty}

\begin{document}\raggedright

\noindent
Name: \rule[-1ex]{0.55\textwidth}{0.1pt}
NetID: \rule[-1ex]{0.2\textwidth}{0.1pt}

\section*{NYU Physics 2---Term Exam 1}

\paragraph{Problem~\theproblem:}\refstepcounter{problem}%
(From Problem Set 1)
How many electrons are there in 1~coulomb of charge?
Give your answer in scientific notation.
If you are having trouble with arithmetic, ask the proctor!

\vfill
\paragraph{Problem~\theproblem:}\refstepcounter{problem}%
(From Recitation, week of 2020-02-03)
Show a dipole in the $x$--$y$ plane, consisting of two charges, $+q$ and $-q$, separated by a
separation $a$ in the $x$-direction, centered on the origin.
What is the direction of the electric field at various locations on the $y$ axis?
Draw a diagram so there is no ambiguity about signs and directions.

\vfill
\paragraph{Problem~\theproblem:}\refstepcounter{problem}%
(From Problem Set 2)
An object has a surface at which there is an electric field discontinuity.
The field is zero inside the surface, and has magnitude $E$ outside the surface
(and direction perpendicular to the surface).
What is the charge density $\sigma$ (charge per unit area) on that surface?
Give an expression for $\sigma$ in terms of $E$, $\epsilon_0$, and anything else you might need.

\vfill
\paragraph{Problem~\theproblem:}\refstepcounter{problem}%
(From Problem Set 2)
How many protons are there in your body, roughly?
Give an answer that is a number in scientific notation (with only one significant digit, say).

\vfill ~
\clearpage

\paragraph{Problem~\theproblem:}\refstepcounter{problem}%
(From Lecture, 2020-02-13)
What, roughly, is the voltage at which high-tension (long distance) power lines run?
Give your answer as a number in volts.

\vfill
\paragraph{Problem~\theproblem:}\refstepcounter{problem}%
(From Problem Set 3)
Imagine you have two plates of very large area $A$ separated by a very small distance $h$.
There is charge $+Q$ on the bottom plate and $-Q$ on the top plate.
What is the field $\vec{E}$ between those plates? Give the magnitude and direction.
Assume that there is nothing else nearby.
Give a symbolic expression in terms of $A$, $h$, $Q$, $\epsilon_0$, and anything else you need.

\vfill
\paragraph{Problem~\theproblem:}\refstepcounter{problem}%
(From Lecture, 2020-02-18)
Why is the electric field at the surface of a charged conductor always
perpendicular to the surface? Answer in 20 words or less. Box your answer.

\vfill
\paragraph{Problem~\theproblem:}\refstepcounter{problem}%
(From Lecture, 2020-02-20)
Imagine you have an isolated charged sphere of radius $a$, holding
charge $Q$ on its surface.
What is the potential difference $V$ between a point on the surface (that
is, at radius $a$) and a point at larger radius $b$?
Give your answer as an expression in terms of $Q$, $a$, $b$, $\epsilon_0$, and
anything else you might need.

\vfill ~
\end{document}
