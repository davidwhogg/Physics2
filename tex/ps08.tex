\documentclass{article}
\begin{document}
\thispagestyle{empty}
\section*{NYU Physics 2 --- Problem Set 8}

\emph{Due by Friday 2002 March 29 at 1pm at Irene Port's office in
Meyer 424.}

\subsection*{Problem 1}

In the first half of this semester, we learned that the electric field
vanishes inside a conductor.  This is only true of a \emph{static}
conductor.  In fact, the charges in a conductor arrange themselves so
they feel \emph{no net force}.  If I move a block of conducting
material at velocity $\vec{v}=v\,\hat{x}$ in a constant magnetic field
$\vec{B}=B\,\hat{y}$, the free charges in the conductor will move to
create an electric field $\vec{E}$ whose electrostatic force exactly
cancels out the magnetic force on the charges in the metal.  What is
the magnitude and direction of this ``induced'' electric field?  If
the block of conductor has a height $Z$ in the $z$ direction, there
will be a potential difference $V$ across the conductor.  How fast $v$
would you have to move a meter-long block of conductor in the magnetic
field of the Earth to get a potential difference of 1~V?  State your
approximations and show your work.

\subsection*{Problem 2}

A rectangular loop of length $X=3~\mathrm{cm}$ and width
$Y=4~\mathrm{cm}$ in the $x$--$y$ plane has a current of
$1~\mathrm{A}$ flowing.  The loop is aligned with the $x$ and $y$ axes
of the coordinate system and one corner is at the origin.  Find the
magnetic field $\vec{B}$ (magnitude and direction) at the point
$(x,y)=(1,1)~\mathrm{cm}$ inside the loop.  Draw a diagram showing the
field direction.  Show the integral you did to get the field and show
how you did it.

\subsection*{Problem 3}

Use Ampere's law to compute the magnetic field $B$ everywhere for an
infinite, thin sheet of current in the $x-y$ plane, moving in the $x$
direction.  This sheet of current will have a current per unit length
$\cal J$ (in that a strip of the sheet of $y$-width $Y$ will contain
current $I= {\cal J}\,Y$.  Before you start the problem, use symmetry
and the right-hand rule to figure out the \emph{only} magnetic field
configuration allowed by the \emph{symmetry} of the problem.

\end{document}
