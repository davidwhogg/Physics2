\documentclass[12pt]{article}
\usepackage{url, graphicx, epstopdf, amsmath, esint}

% page layout
\setlength{\topmargin}{-0.25in}
\setlength{\textheight}{9.5in}
\setlength{\headheight}{0in}
\setlength{\headsep}{0in}
\setlength{\parindent}{1.1\baselineskip}
\addtolength{\oddsidemargin}{-0.75in}
\setlength{\marginparwidth}{2in}

% problem formatting
\newcommand{\problemname}{Problem}
\newcounter{problem}
\newcommand{\startproblem}{\paragraph{Problem~\theproblem:}\refstepcounter{problem}}

% words
\newcommand{\foreign}[1]{\textsl{#1}}
\newcommand{\vs}{\foreign{vs}}

% math
\renewcommand{\vec}[1]{\boldsymbol{#1}}
\newcommand{\dd}{\mathrm{d}}
\newcommand{\e}{\mathrm{e}}
\newcommand{\cross}{\times}
\newcommand{\curl}{\vec{\nabla}\times}

% primary units
\newcommand{\rad}{\mathrm{rad}}
\newcommand{\kg}{\mathrm{kg}}
\newcommand{\m}{\mathrm{m}}
\newcommand{\s}{\mathrm{s}}
\newcommand{\A}{\mathrm{A}}

% secondary units
\renewcommand{\deg}{\mathrm{deg}}
\newcommand{\km}{\mathrm{km}}
\newcommand{\cm}{\mathrm{cm}}
\newcommand{\mm}{\mathrm{mm}}
\newcommand{\mum}{\mathrm{\mu m}}
\newcommand{\nm}{\mathrm{nm}}
\newcommand{\ft}{\mathrm{ft}}
\newcommand{\mi}{\mathrm{mi}}
\newcommand{\AU}{\mathrm{AU}}
\newcommand{\ns}{\mathrm{ns}}
\newcommand{\h}{\mathrm{h}}
\newcommand{\yr}{\mathrm{yr}}
\newcommand{\N}{\mathrm{N}}
\newcommand{\J}{\mathrm{J}}
\newcommand{\eV}{\mathrm{eV}}
\newcommand{\MeV}{\mathrm{MeV}}
\newcommand{\W}{\mathrm{W}}
\newcommand{\Pa}{\mathrm{Pa}}
\newcommand{\C}{\mathrm{C}}
\newcommand{\V}{\mathrm{V}}
\newcommand{\ohm}{\mathrm{\Omega}}
\newcommand{\muF}{\mathrm{\mu F}}
\newcommand{\Hz}{\mathrm{Hz}}
\newcommand{\GHz}{\mathrm{GHz}}

% derived units
\newcommand{\mps}{\m\,\s^{-1}}
\newcommand{\mph}{\mi\,\h^{-1}}
\newcommand{\mpss}{\m\,\s^{-2}}
\newcommand{\radps}{\rad\,\s^{-1}}

% random stuff
\sloppy\sloppypar\raggedbottom\frenchspacing\thispagestyle{empty}

\begin{document}

\section*{NYU Physics 2---Problem Set 2}

Due Thursday 2020 February 13 at the beginning of lecture.

\paragraph{Problem~\theproblem:}\refstepcounter{problem}%
Imagine a rod with length $\pi\,R$ and uniformly distributed charge
$Q$ bent into a semi-circle of radius $R$.  What is the direction and
magnitude of the electric field $\vec{E}$ at the center of the
semi-circle?

\paragraph{Problem~\theproblem:}\refstepcounter{problem}%
Imagine that the charge on the proton is bigger in magnitude than the
charge on the electron by 1 part in a trillion ($10^{12}$).  What is
the magnitude of the electrostatic force between you and a friend who
is $1~\mathrm{m}$ away?  How much total work did the two of you have
to do to get that close to one another (assuming you started very far
apart)? You might need to look up electrostatic energy to solve that last part.
And you might need to make the spherical-student approximation!

\paragraph{Problem~\theproblem:}\refstepcounter{problem}%
A thin spherical shell of radius $R$ contains a uniformly distributed
charge $Q$.  Imagine that this is in space with nothing else near
it. Use Gauss's Law to get the electric field everywhere. It will be
different in form at $r<R$ and at $r>R$. There is an Electric field
discontinuity at $r=R$ (between the inside and the outside of the sphere). Show
that the change in the electric field as you cross the shell is equal
to $\sigma/\epsilon_0$, where $\sigma$ is the charge per unit area on
the shell.

\paragraph{Problem~\theproblem:}\refstepcounter{problem}%
A very long (ie, infinite) cylinder of radius $a$ is filled with a
constant positive charge density $\rho$ (charge per unit volume).
Concentric with this cylinder is a thin (ie, zero thickness) outer
uniformly negatively charged cylindrical shell of radius $b>a$, with
equal and opposite total charge (or charge per unit length) to that
contained in the inner cylinder.  There are no other charges anywhere near.  Use
Gauss's law to determine the magnitude $E$ of the electric field at
any perpendicular distance $r$ from the center of the cylinders.
Write explicit expressions for $E$ in the three regions $r<a$, $a<r<b$
and $r>b$.  Make a plot of $E(r)$, noting the quantitative values of
$E$ at $r=0$, $r=a$, $r=b$, and $r=\infty$.  Are there any
discontinuities in $E$?  If so, where?  What is the surface charge
density $\sigma$ (charge per unit area) on the outer cylindrical
shell?

\end{document}
